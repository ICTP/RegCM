%%
%%   This file is part of ICTP RegCM.
%%
%%   ICTP RegCM is free software: you can redistribute it and/or modify
%%   it under the terms of the GNU General Public License as published by
%%   the Free Software Foundation, either version 3 of the License, or
%%   (at your option) any later version.
%%
%%   ICTP RegCM is distributed in the hope that it will be useful,
%%   but WITHOUT ANY WARRANTY; without even the implied warranty of
%%   MERCHANTABILITY or FITNESS FOR A PARTICULAR PURPOSE.  See the
%%   GNU General Public License for more details.
%%
%%   You should have received a copy of the GNU General Public License
%%   along with ICTP RegCM.  If not, see <http://www.gnu.org/licenses/>.
%%

\chapter{Future Developments}

We have lot of exciting plans for future model improvements, some of which
are in a already mature stage and under testing, with some published results,
whereas others are done only on the paper in a whishlist for next years.
Nevertheless we want to share this with users, to have hints and encourage
contributions.
Some of the development results/ideas are listed below, in a "time to market"
order.

\section{UFW PBL scheme}

One of the deficiencies identified in \ac{RegCM3} has been the lack of
simulation of low level stratus clouds, a problem clearly tied to the excessive
vertical transport in the Holtslag PBL scheme (\cite{OBrien_11}).
To address this problem Travis O’Brien coupled to the \ac{RegCM4} the general
turbulence closure parameterization of \citep{Grenier_01,Bretherton_04},
which we refer to as UW-PBL. This is a 1.5 order local, down-gradient diffusion
parameterization in which the velocity scale is based on turbulent kinetic
energy (TKE). The TKE is in turn calculated prognostically from the balance of
buoyant production/destruction, shear production, dissipation vertical transport
and horizontal diffusion and advection. The scheme also parameterizes the
entrainment process  and its enhancement by evaporation of cloudy air into
entrained air. One property of the scheme is the use of a mixing length
formulation based on a 2010 paper by Grisogono (ref?) which allows a more
realistic description of sharp inversions under strongly stable conditions.
The UW-PBL has been so far tested within the \ac{RegCM4} framework mostly in
midlatitude domains, such as the continental US (where it considerably improved
the simulation of low level stratus clouds) and Europe.

This scheme is currently in a SVN branch of the code and will be merged into
the main development trunk as soon as the accompanying paper will be published,
and will be available in the next model release.

\section{Tiedtke convection scheme}

Adrian Tompkins is developing an adaptation of the ECHAM5.4 \cite{Tiedtke_89}
cumulus convection scheme for the \ac{RegCM} model. The code from ECHAM has
been ported into RegCM, and extensive testing is planned in the second half of
2011. This option should be available for next model release.

\section{Chemistry}

Fabien Solmon is developing the coupling of \ac{RegCM} model with the CBMZ
chemical module with the Sillmann fast solver.

\section{Coupling}

We have resolved to adopt for the \ac{RegCM} model a standard model coupling
engine: the \ac{ESMF}. Ufuk Utku Turuncoglu is already adapting model data
structures to use the \ac{ESMF} framework. First target will be to couple
the \ac{RegCM} model to the \ac{ROMS} oceanic model, and update the \ac{CLM}
to version 4.

\section{2D parallelization}

This long standing limitation of the model in the parallel performances will
be faced: we plan to drop altogether the Serial model version (does exist
anymore a single core processor?), clean up model parallel code and perform
a dynamical 2D decomposition of the model domain.

\section{Parallel I/O}

This is another limit of the current model implementation, where all data
need to be gathered by the master processor before being written to disk.
Again, if running on a decent cluster, all processors usually have access
to disk resources, and a form of parallel I/O will allow a big performance
boost as well as a reduction of some of the MPI communication data at the
expenses of an increase of the requirements for the cluster I/O channel.

\section{Semi-Lagrangian dynamic core}

A semi-Lagrangian advection scheme for the water vapor and advection tracers
will allow a different timestep for the transport schemes, which should result
in a performance prize.

\section{Non-Hydrostatic core}

We want to implement the non-hydrostatic core to allow physical downscaling
of large scale model simulation under the 20 kilometers limit of the
hydrostatic model.
