%
% This file is part of ICTP RegCM model.
% Copyright (C) 2011 ICTP Trieste
% See the file COPYING for copying conditions.
%

The Regional Climate Model developed at ICTP has now reached its 4.5 release.
The code base is actively developed by a community of developers internal and
external to ICTP, and this work is merged on the
Gforge site on gforge.ictp.it site.

The main new technical features of the release 4.5 are summarized in the
following points:

\begin{itemize}
  \item The MM5 non-hydrostatic dynamical core has been ported to the ICTP
	  RegCM without removing the existing hydrostatic core. Now the user
	  can select in the input namelist which of the two dynamical cores
	  he wish to use.
  \item The SAV file strategy has changed to allow a better integration of
	  the RegCM into the RegESM framework and the user is required to
	  modify the setting from previous version namelist files.
  \item The source for the topography is now the GMTED dataset which is an
	  update to the USGS GTOPO dataset.
  \item The DUST 12 bin option has been added to the aerosol simulation types
  \item The chemistry/aerosol option now works along the CLM4.5 option
  \item The model supports soil moisture initialization from a previous run
	  for both the BATS and the CLM4.5 surface models.
\end{itemize}
Bug Fixing:
\begin{itemize}
  \item Bugs present in the CN schemes for the CLM4.5 implemented in RegCM
	  have been fixed
  \item A problematic interaction between the UW pbl and the new Microphisics
	  has been fixed
\end{itemize}

Next release V 5.X :

\begin{itemize}
  \item New dynamical core from Giovanni Tumolo semi-implicit, semi-Lagrangian,
   p-adaptive Discontinuous Galerkin method three dimensional model.
\end{itemize}

The model code is in Fortran 2003 ANSI standard.
The development is done on Linux boxes, and the model is known to run
on IBM AIX\texttrademark platforms, MacOS\texttrademark platforms.
No porting effort has been done towards non Unix-like Operating Systems.
We will for this User Guide assume that the reference platform is a recent
Linux distributon with a \verb=bash= shell.
Typographical convention is the following:

\begin{table}[ht]
\caption{Conventions}
\vspace{0.05 in}
\centering
\begin{tabular}{l|l}
\hline
\verb=$> = & normal shell prompt \\
\verb=#> = & root shell prompt \\
\verb=$SHELL_VARIABLE = & a shell variable \\
\hline
\end{tabular}
\label{conventions}
\end{table}

Any shell variable is supposed to be set by the User with the following example
syntax:

\begin{Verbatim}
$> export REGCM_ROOT="/home/user/RegCM-4.5.0"
\end{Verbatim}

Hope you will find this document useful. Any error found belongs to the
RegCNET and can be reported to be corrected in future revisions. Enjoy.
