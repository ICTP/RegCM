%
% This file is part of ICTP RegCM model.
% Copyright (C) 2024 ICTP Trieste
% See the file COPYING for copying conditions.
%

The Regional Climate Model developed at ICTP has now reached its 5.0 release.
The code base is actively developed by a community of developers internal and
external to ICTP, and this work is merged on the GitHub site.

The main new technical features of the release 5.0 are summarized in the
following points:

\begin{itemize}
  \item A new fast non-hydrostatical core derived from CNR MOLOCH has
      been added, and now RegCM has three different dynamical cores.
  \item The CMIP6 Input4MIPS dataset is now used by the RegCM in its
      original format, and climatological (MERRA2) and Simple Plume model
      data can be used as non interactive aerosol for the radiation code.
  \item The ERA5 and CMIP6 data from a number of different GCMs are now
      supported.
\end{itemize}
For the list of fixed bug, please refer to the git log.

The model code is in Fortran 2003 ANSI standard.
The development is done on Linux boxes and porting effort will be invested
toward porting the model on non Unix-like Operating Systems.
We will for this User Guide assume that the reference platform is a recent
Linux distributon with a \verb=bash= shell.
Typographical convention is the following:

\begin{table}[ht]
\caption{Conventions}
\vspace{0.05 in}
\centering
\begin{tabular}{l|l}
\hline
\verb=$> = & normal shell prompt \\
\verb=#> = & root shell prompt \\
\verb=$SHELL_VARIABLE = & a shell variable \\
\hline
\end{tabular}
\label{conventions}
\end{table}

Any shell variable is supposed to be set by the User with the following example
syntax:

\begin{Verbatim}
$> export REGCM_ROOT="/home/user/RegCM"
\end{Verbatim}

Hope you will find this document useful. Any error found belongs to the
RegCNET and can be reported to be corrected in future revisions. Enjoy.
